\documentclass[12pt]{article}

\usepackage[a4paper, top=25mm, right=30mm, bottom=25mm, left=30mm]{geometry}
\usepackage[russian]{babel}
\usepackage{fontspec}
\usepackage{graphicx}
\usepackage[unicode]{hyperref}

\setmainfont{Times New Roman}
\graphicspath{{images/}}

\title{Lab6 OPD}
\author{Азат Сиразетдинов}

\begin{document}
	\thispagestyle{empty}
	\begin{center}
		Федеральное государственное автономное образовательное учреждение\\ 
		высшего образования\\
		«Национальный исследовательский университет ИТМО»\\
		\textit{Факультет Программной Инженерии и Компьютерной Техники}\\
	\end{center}
	\vspace{2cm}
	\begin{center}
		\large
		\textbf{Лабораторная работа № 6}\\
		по дисциплине ОПД\\
		Обмен по прерыванию\\
		Вариант № 1323
	\end{center}
	\vspace{7cm}
	\begin{flushright}
		Выполнил:\\
		cтудент  группы P3116\\
		Сиразетдинов А. Н\\
		Преподаватель: \\
		Афанасьев Д. Б.\\
	\end{flushright}
	\vspace{6cm}
	\begin{center}
		г. Санкт-Петербург\\
		2022г.
	\end{center}
	\newpage
	
	\tableofcontents
	
	\newpage
	\section{Задание}
	
	По выданному преподавателем варианту разработать и исследовать работу комплекса программ обмена данными в режиме прерывания программы. Основная программа должна изменять содержимое заданной ячейки памяти (Х), которое должно быть представлено как знаковое число. Область допустимых значений изменения Х должна быть ограничена заданной функцией F(X) и конструктивными особенностями регистра данных ВУ (8-ми битное знаковое представление). Программа обработки прерывания должна выводить на ВУ модифицированное значение Х в соответствии с вариантом задания, а также игнорировать все необрабатываемые прерывания.
	\begin{enumerate}
		\item Основная программа должна уменьшать на 3 содержимое X (ячейки памяти с адресом 03916) в цикле.
		\item Обработчик прерывания должен по нажатию кнопки готовности ВУ-3 осуществлять вывод результата вычисления функции F(X)=3X+6 на данное ВУ, a по нажатию кнопки готовности ВУ-2 изменить знак содержимого РД данного ВУ и записать в Х
		\item Если Х оказывается вне ОДЗ при выполнении любой операции по его изменению, то необходимо в Х записать максимальное по ОДЗ число.
	\end{enumerate}
\end{document}